%\begin{center}
%	\begin{tikzpicture}
%		\begin{axis}[thick,smooth,no markers, hide axis, ymin=-3, ymax=2, xmin=-1, xmax=5]
%			\draw[thick, arrows=<-] (axis cs:0, 2) -- (axis cs:0, -3);
%			\draw[thick, arrows=->] (axis cs:-1, 0) -- (axis cs:5, 0);
%			\node[text, below left] at (axis cs:0.0, 0.0) {$O$};
%			\draw [cyan, samples=100] plot [smooth, tension=0.75] coordinates { (axis cs:-1,-3) (axis cs:0,-1) (axis cs:1,0) (axis cs:2,-1) (axis cs:3,-2) (axis cs:4,0) (axis cs:5,2) (axis cs:6,0) };
%		\end{axis}
%	\end{tikzpicture}
%\end{center}

\subsection{Part A}
1, 3. This is because $f'(x)$ is zero (has a horizontal tangent) at these locations.

\subsection{Part B}
Min: $x = 4$, the graph had been negative until then, meaning that $f(x)$ decreased until this point. \\ \\
Max: $x = -1$, the graph stays negative for more of the graph between $-1 \leq x \leq 5$ than it stays positive. This means that there is no higher value of $f(x)$ than the beginning.

\subsection{Part C}
\[ g(x) = xf(x) \]
\[ g'(x) = (1)f(x) + xf'(x) \]
\[ m = g'(2) = (1)f(2) + xf'(2) = 6 + (2)(-1) = 6 - 2 = 4 \]
\[ y = f(2) = 6, \enspace x = 2 \]
\[ y - 6 = 4 (x - 2) \]
